\documentclass[12pt,a4paper]{article}
\usepackage[utf8x]{inputenc}
\usepackage[T2A]{fontenc} 
\usepackage{ucs}
\usepackage[russian,english]{babel}
\usepackage{amsmath}
\usepackage{amsfonts}
\usepackage{amssymb}
\usepackage{listings}
\usepackage{color}
\usepackage{svninfo}
%\svnInfo $Id$
\author{Куренский Евгений Александрович}
\newcommand{\Track}{$\textit{Track}^{\textsc{\textcopyright}}$}

\title{Инструкция по установке и настройке системы \Track{}}

\begin{document}
\selectlanguage{russian}
\maketitle
\newpage
\tableofcontents
\newpage
\begin{abstract}
Система \Track{} предназначена для сбора и обработки данных датчиков установленных на оборудовании для последующего представления и анализа. Активно используется информация ГИС для построения трэков движения и привязки к дорогам.

Документ описывает требования к программному обеспечению (ПО) и процедуру установки данного программного комплекса.
\end{abstract}
\newpage
\section{Возможности подсистемы \Track{}}
\begin{enumerate}
\item Сбор и хранение журналов работы модулей в реляционной БД.
\item Ведение справочной информации по клиентам.
\item Ведение справочной информации по объектам.
\item Ведение справочной информации ГИС.
\item Интеграция системы с Google Maps.
\item Клинетская авторизация.
\item Детальное разграничение доступа на уровне строк.
\item Отчёт о состоянии объектов.
\item Отчёт о пройденном пути за указанный интервал дат.
\item Время работы и простоя устройства.
\end{enumerate}
\section{Аппаратные требования к системе}
\subsection{Требование к серверной платформе}
Данные требования предполагают учёт нагрузки и увеличения количества устройств в 2 раза по сравнению с текущим количеством устройств.
\begin{enumerate}
\item Процессор: 2 процессора Intel x86-x86\_64 (или подобный по производительности) с тактовой частотой не менее 2 Гигарц.
\item ОЗУ: не менее 1 Гигабайта.
\item Дисковый массив: не менее 2 SCSI дисков объемом не менее 120 Гбайт при текущем количестве устройств.
\item Сетевые устройства: сетевая карта Ethernet не менее 10 мБит.
\item Порты ввода-вывода: 2 порта RS232, не менее 1 порта USB.
\item Периферийые устройства: Источник бесперебойного питания Smart APC мощьностью не менее 1 кВт/А.
\end{enumerate}
\subsection{Требование к настройке программного комплекса}\label{lab1}
\begin{enumerate}
\item Операционная система: Linux
\item Сетевой интерфейс: Статический адрес из диапазона Internet
\item БД: \textbf{Postgres DB версии 8.1} и выше, \textbf{PostGIS 1.2} и выше, добавлена поддержка языков \textbf{Pgplsql} и \textbf{Pgperl}.
\item \textbf{Perl:} версии 5.8 выше, модули \textbf{Net::Telnet, Net::Server::Fork, DBI::Pg}
\item \textbf{Apache:} версии 2.0 и выше + \textbf{php\_mod} 
\item \textbf{PHP:} версии 5
\end{enumerate}
\subsection{Порядок установки и настройки системы}
Предполагается что предварительно произведена установка и настройка операционной системы согласно требованиям \ref{lab1}

Порядок установки:
\begin{enumerate}
\item Создать БД с кодировкой UTF-8.
\item Создать 2 табличных пространства \textbf{user\_data} для данных и \\ \textbf{user\_ind} для индексов. Рекомендуется размещать данные табличные пространства на разных физических дисках для увеличения производительности БД. 
Пример создание БД и генерации табличных пространств находится в файле \textit{dbtrack.db} из поставки.
\item Создать схему \textbf{owner\_track} и пользователя \textbf{owner\_track} запустив на исполнение под админстратором БД скрипт \textit{dbtrack.us} из поставки.
\item Создать необходимые роли в БД запустив на исполнение под админстратором БД скрипт \textit{dbtrack.rg} из поставки.
\item Создать таблицы запустив на исполнение под \textbf{owner\_track} скрипт \textit{trackddl.sql} из поставки.
\item Заполнить системные справочники запустив на исполнение под \textbf{owner\_track} скрипт \textit{instantiate.sql} из поставки.
\item Заполнить справочники районов НСО запустив на исполнение под \textbf{owner\_track} скрипт \textit{insert\_area.sql} из поставки.
\item Создать процедуры PGPERL запустив на исполнение под \textbf{owner\_track} скрипт \textit{functions\_perl.sql} из поставки.
\item Создать процедуры PGPLSQL запустив на исполнение под \textbf{owner\_track} скрипт \textit{functions\_pgplsql.sql} из поставки.
\item Создать представления запустив на исполнение под \textbf{owner\_track} скрипт \textit{views.sql} из поставки.

\end{enumerate}

Дальнейшие настройки выполняются из пользовательского интерфейса.

\end{document}
